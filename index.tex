% Options for packages loaded elsewhere
\PassOptionsToPackage{unicode}{hyperref}
\PassOptionsToPackage{hyphens}{url}
\PassOptionsToPackage{dvipsnames,svgnames,x11names}{xcolor}
%
\documentclass[
  letterpaper,
  DIV=11,
  numbers=noendperiod]{scrreprt}

\usepackage{amsmath,amssymb}
\usepackage{iftex}
\ifPDFTeX
  \usepackage[T1]{fontenc}
  \usepackage[utf8]{inputenc}
  \usepackage{textcomp} % provide euro and other symbols
\else % if luatex or xetex
  \usepackage{unicode-math}
  \defaultfontfeatures{Scale=MatchLowercase}
  \defaultfontfeatures[\rmfamily]{Ligatures=TeX,Scale=1}
\fi
\usepackage{lmodern}
\ifPDFTeX\else  
    % xetex/luatex font selection
\fi
% Use upquote if available, for straight quotes in verbatim environments
\IfFileExists{upquote.sty}{\usepackage{upquote}}{}
\IfFileExists{microtype.sty}{% use microtype if available
  \usepackage[]{microtype}
  \UseMicrotypeSet[protrusion]{basicmath} % disable protrusion for tt fonts
}{}
\makeatletter
\@ifundefined{KOMAClassName}{% if non-KOMA class
  \IfFileExists{parskip.sty}{%
    \usepackage{parskip}
  }{% else
    \setlength{\parindent}{0pt}
    \setlength{\parskip}{6pt plus 2pt minus 1pt}}
}{% if KOMA class
  \KOMAoptions{parskip=half}}
\makeatother
\usepackage{xcolor}
\setlength{\emergencystretch}{3em} % prevent overfull lines
\setcounter{secnumdepth}{5}
% Make \paragraph and \subparagraph free-standing
\ifx\paragraph\undefined\else
  \let\oldparagraph\paragraph
  \renewcommand{\paragraph}[1]{\oldparagraph{#1}\mbox{}}
\fi
\ifx\subparagraph\undefined\else
  \let\oldsubparagraph\subparagraph
  \renewcommand{\subparagraph}[1]{\oldsubparagraph{#1}\mbox{}}
\fi


\providecommand{\tightlist}{%
  \setlength{\itemsep}{0pt}\setlength{\parskip}{0pt}}\usepackage{longtable,booktabs,array}
\usepackage{calc} % for calculating minipage widths
% Correct order of tables after \paragraph or \subparagraph
\usepackage{etoolbox}
\makeatletter
\patchcmd\longtable{\par}{\if@noskipsec\mbox{}\fi\par}{}{}
\makeatother
% Allow footnotes in longtable head/foot
\IfFileExists{footnotehyper.sty}{\usepackage{footnotehyper}}{\usepackage{footnote}}
\makesavenoteenv{longtable}
\usepackage{graphicx}
\makeatletter
\def\maxwidth{\ifdim\Gin@nat@width>\linewidth\linewidth\else\Gin@nat@width\fi}
\def\maxheight{\ifdim\Gin@nat@height>\textheight\textheight\else\Gin@nat@height\fi}
\makeatother
% Scale images if necessary, so that they will not overflow the page
% margins by default, and it is still possible to overwrite the defaults
% using explicit options in \includegraphics[width, height, ...]{}
\setkeys{Gin}{width=\maxwidth,height=\maxheight,keepaspectratio}
% Set default figure placement to htbp
\makeatletter
\def\fps@figure{htbp}
\makeatother
% definitions for citeproc citations
\NewDocumentCommand\citeproctext{}{}
\NewDocumentCommand\citeproc{mm}{%
  \begingroup\def\citeproctext{#2}\cite{#1}\endgroup}
\makeatletter
 % allow citations to break across lines
 \let\@cite@ofmt\@firstofone
 % avoid brackets around text for \cite:
 \def\@biblabel#1{}
 \def\@cite#1#2{{#1\if@tempswa , #2\fi}}
\makeatother
\newlength{\cslhangindent}
\setlength{\cslhangindent}{1.5em}
\newlength{\csllabelwidth}
\setlength{\csllabelwidth}{3em}
\newenvironment{CSLReferences}[2] % #1 hanging-indent, #2 entry-spacing
 {\begin{list}{}{%
  \setlength{\itemindent}{0pt}
  \setlength{\leftmargin}{0pt}
  \setlength{\parsep}{0pt}
  % turn on hanging indent if param 1 is 1
  \ifodd #1
   \setlength{\leftmargin}{\cslhangindent}
   \setlength{\itemindent}{-1\cslhangindent}
  \fi
  % set entry spacing
  \setlength{\itemsep}{#2\baselineskip}}}
 {\end{list}}
\usepackage{calc}
\newcommand{\CSLBlock}[1]{\hfill\break\parbox[t]{\linewidth}{\strut\ignorespaces#1\strut}}
\newcommand{\CSLLeftMargin}[1]{\parbox[t]{\csllabelwidth}{\strut#1\strut}}
\newcommand{\CSLRightInline}[1]{\parbox[t]{\linewidth - \csllabelwidth}{\strut#1\strut}}
\newcommand{\CSLIndent}[1]{\hspace{\cslhangindent}#1}

\KOMAoption{captions}{tableheading}
\makeatletter
\@ifpackageloaded{tcolorbox}{}{\usepackage[skins,breakable]{tcolorbox}}
\@ifpackageloaded{fontawesome5}{}{\usepackage{fontawesome5}}
\definecolor{quarto-callout-color}{HTML}{909090}
\definecolor{quarto-callout-note-color}{HTML}{0758E5}
\definecolor{quarto-callout-important-color}{HTML}{CC1914}
\definecolor{quarto-callout-warning-color}{HTML}{EB9113}
\definecolor{quarto-callout-tip-color}{HTML}{00A047}
\definecolor{quarto-callout-caution-color}{HTML}{FC5300}
\definecolor{quarto-callout-color-frame}{HTML}{acacac}
\definecolor{quarto-callout-note-color-frame}{HTML}{4582ec}
\definecolor{quarto-callout-important-color-frame}{HTML}{d9534f}
\definecolor{quarto-callout-warning-color-frame}{HTML}{f0ad4e}
\definecolor{quarto-callout-tip-color-frame}{HTML}{02b875}
\definecolor{quarto-callout-caution-color-frame}{HTML}{fd7e14}
\makeatother
\makeatletter
\@ifpackageloaded{bookmark}{}{\usepackage{bookmark}}
\makeatother
\makeatletter
\@ifpackageloaded{caption}{}{\usepackage{caption}}
\AtBeginDocument{%
\ifdefined\contentsname
  \renewcommand*\contentsname{Índice}
\else
  \newcommand\contentsname{Índice}
\fi
\ifdefined\listfigurename
  \renewcommand*\listfigurename{Lista de Figuras}
\else
  \newcommand\listfigurename{Lista de Figuras}
\fi
\ifdefined\listtablename
  \renewcommand*\listtablename{Lista de Tabelas}
\else
  \newcommand\listtablename{Lista de Tabelas}
\fi
\ifdefined\figurename
  \renewcommand*\figurename{Figura}
\else
  \newcommand\figurename{Figura}
\fi
\ifdefined\tablename
  \renewcommand*\tablename{Tabela}
\else
  \newcommand\tablename{Tabela}
\fi
}
\@ifpackageloaded{float}{}{\usepackage{float}}
\floatstyle{ruled}
\@ifundefined{c@chapter}{\newfloat{codelisting}{h}{lop}}{\newfloat{codelisting}{h}{lop}[chapter]}
\floatname{codelisting}{Listagem}
\newcommand*\listoflistings{\listof{codelisting}{Lista de Listagens}}
\makeatother
\makeatletter
\makeatother
\makeatletter
\@ifpackageloaded{caption}{}{\usepackage{caption}}
\@ifpackageloaded{subcaption}{}{\usepackage{subcaption}}
\makeatother
\ifLuaTeX
\usepackage[bidi=basic]{babel}
\else
\usepackage[bidi=default]{babel}
\fi
\babelprovide[main,import]{portuguese}
% get rid of language-specific shorthands (see #6817):
\let\LanguageShortHands\languageshorthands
\def\languageshorthands#1{}
\ifLuaTeX
  \usepackage{selnolig}  % disable illegal ligatures
\fi
\usepackage{bookmark}

\IfFileExists{xurl.sty}{\usepackage{xurl}}{} % add URL line breaks if available
\urlstyle{same} % disable monospaced font for URLs
\hypersetup{
  pdftitle={Aulas},
  pdfauthor={Norah Jones},
  pdflang={pt},
  colorlinks=true,
  linkcolor={blue},
  filecolor={Maroon},
  citecolor={Blue},
  urlcolor={Blue},
  pdfcreator={LaTeX via pandoc}}

\title{Aulas}
\author{Norah Jones}
\date{2024-10-02}

\begin{document}
\maketitle

\renewcommand*\contentsname{Índice}
{
\hypersetup{linkcolor=}
\setcounter{tocdepth}{2}
\tableofcontents
}
\bookmarksetup{startatroot}

\chapter*{Preface}\label{preface}
\addcontentsline{toc}{chapter}{Preface}

\markboth{Preface}{Preface}

This is a Quarto book.

To learn more about Quarto books visit
\url{https://quarto.org/docs/books}.

\bookmarksetup{startatroot}

\chapter{Introdução: Problemas de
interesse}\label{introduuxe7uxe3o-problemas-de-interesse}

\begin{itemize}
\item
  Encontrar soluções de equações não lineares onde não é possível obter
  uma solução analítica;
\item
  Obter integrais que apresentam uma forma complicada que inviabiliza
  encontrar uma solução analítica;
\item
  Gerar artificialmente amostras a partir de modelos estatísticos;
\item
  Aplicar a metodologia estudada na resolução de problemas de
  inferência.
\end{itemize}

\part{Solução Numérica de Equações}

\section*{Motivação - O Estimador de Máxima
Verossimilhança}\label{motivauxe7uxe3o---o-estimador-de-muxe1xima-verossimilhanuxe7a}
\addcontentsline{toc}{section}{Motivação - O Estimador de Máxima
Verossimilhança}

\markright{Motivação - O Estimador de Máxima Verossimilhança}

No que segue o termo densidade, significa ou uma densidade de
probabilidade (caso absolutamente contínuo) ou uma função de
probabilidade (caso discreto).

Sejam
\(X_1, \  \dots, \ X_n \overset{iid}{\sim} f(.|\theta), \ \theta \in \ \Theta\),
onde \(f(.|\theta)\) é uma densidade, \(\theta\) é um parâmetro que
desejamos estimar e \(\Theta\) é o espaço paramétrico;

Suponha que observamos os valores \(x_1, \ \dots, \ x_n\). A função de
verossimilhança é definida por:

\[ L(\theta) = \prod_{i=1}^{n} f(x_i|\theta), \ \theta \in \Theta \]

A função de log-verossimilhança é dada por:

\[ l(\theta) = logL(\theta) = \sum_{i=1}^{n} log f(x_i|\theta), \ \theta \ \in \Theta\]

Seja \(\hat{\theta} \ \in \Theta\) um valor do parâmetro que maximiza a
função de verossimilhança, ou seja, tal que

\[L(\hat{\theta}) \ \geq L(\hat{\theta}), \ \text{para todo} \ \theta \ \in \ \Theta\]

Então dizemos que \(\hat{\theta}\) é uma estimativa de máxima
verossimilhança de \(\theta\).

A interpretação no caso discreto: é mais provável que \(\hat{\theta}\)
tenha gerado os dados \(x_1, \  \dots, \ x_n\)

Como \(\hat{\theta}\) depende da amostra, escrevemos
\(\hat{\theta}(x_1, \ \dots, x_n)\). Neste caso,
\(\hat{\theta}(X_1, \ \dots, \ X_n)\) é o estimador de máxima
verossimilhança (EMV).

\begin{tcolorbox}[enhanced jigsaw, arc=.35mm, toptitle=1mm, coltitle=black, opacityback=0, opacitybacktitle=0.6, toprule=.15mm, colframe=quarto-callout-note-color-frame, titlerule=0mm, bottomtitle=1mm, breakable, colbacktitle=quarto-callout-note-color!10!white, rightrule=.15mm, leftrule=.75mm, colback=white, left=2mm, title=\textcolor{quarto-callout-note-color}{\faInfo}\hspace{0.5em}{Nota}, bottomrule=.15mm]

\begin{itemize}
\item
  Para cada amostra observada \(\textbf{x} = \ (x_1, \ \dots, x_n)\)
\item
  A definição nos diz que \(\hat{\theta}(\textbf{x})\) é um ponto de
  máximo global. Podemos ter nenhum ou mais de um máximo global
\item
  Suponha que \(\Theta\) é um intervalo e que o ponto \(\hat{\theta}\) é
  um ponto interior de \(\Theta\) que é ponto de máximo de L, podendo
  ser um máximo local. Se L tem derivada em \(\hat{\theta}\), então
  \(L'( \hat{\theta}) = 0\). Ou seja, \(\hat{\theta}\) é um ponto
  estacionário de L (também dizemos \(\hat{\theta}\) é um zero da função
  L'). Este resultado é conhecido no Cálculo como Teorema de Fermat para
  Pontos Estacionários.
\item
  Ou seja, sob as condições acima, se \(\hat{\theta}\) for EMV, então a
  derivada de L se anula neste ponto. A recíproca pode não ser
  verdadeira.
\end{itemize}

\end{tcolorbox}

Assim, em muitos casos encontrados na prática, encontrar o EMV é um
problema relacionado a encontrar soluções em \(\theta\) para a equação
\(L'(\hat{\theta}) = 0\) ou \(l'(\theta) = 0\).

Por exemplo, considere uma amostra aleatória \(X_1, \dots, X_n\)
proveniente de uma distribuição \(exp(\theta)\). Assim, cada \(X_i\) tem
densidade de probabilidade.

\[
    f(x|\theta) = \left\{
    \begin{matrix}
        \theta \exp(-\theta x), &  \mbox{se} & x>0\\
          0, &  \mbox{se} & x\leq 0. 
    \end{matrix}
         \right.
\]

O espaço paramétrico é \(\Theta = (0, \infty)\). Temos então a seguinte
função de log-verassimilhança: \[
\ell(\theta)= n \log \theta -\theta \sum_{i=1}^n x_i, \,\,\, \theta > 0, 
\]

implicando em

\[
\ell'(\theta)= n/\theta - \sum_{i=1}^n x_i, \,\,\, \theta > 0. 
\]

A solução da equação \(\ell'(\theta)=0\) é dada por
\(\widehat{\theta}= n/\sum_{i=1}^n x_i\)

A função \(S(\theta)=\ell'(\theta)\), \(\theta \in \Theta\), é
denominada \emph{função escore}. Assim, em geral, encontrar o EMV é
equivalente a resolver em \(\theta\) a equação \(S(\theta)=0\).

No exemplo acima obtivemos uma solução analítica para esta equação. Mas
em algumas situações isto não é possível.

Considere o seguinte exemplo (Bolfarine \& Sandoval, 2010): sejam
\(X_1, \ldots,X_n \overset{iid}{\sim} f(.|\theta)\), onde

\[
    f(x|\theta) = \left\{
    \begin{matrix}
        \frac{1}{2}(1+\theta x), &  \mbox{se} & x>0\\
          0, &  \mbox{se} & x\leq 0. 
    \end{matrix}
         \right.
\]

Onde \(-1 < \theta <1\) verifique que \(f(\cdot|\theta)\) é uma
densidade.

Então é possível mostrar que \[
S(\theta)= \sum_{i=1}^n \frac{x_i}{1+\theta x_i}. 
\]

No entanto, não há solução analítica para \(S(\theta)=0\). Os gráficos a
seguir mostram o comportamento das funções de log-verossimilhança (à
esquerda) e escore (à direita). A amostra utilizada foi gerada
artificialmente a partir do modelo , com \(n=40\) e \(\theta=-0,35\)
(como gerar esta amostra? voltaremos em breve a este assunto).

\chapter{Método da Bisseção}\label{muxe9todo-da-bisseuxe7uxe3o}

asdajhduahd

\bookmarksetup{startatroot}

\chapter{Summary}\label{summary}

In summary, this book has no content whatsoever.

\bookmarksetup{startatroot}

\chapter*{References}\label{references}
\addcontentsline{toc}{chapter}{References}

\markboth{References}{References}

\phantomsection\label{refs}
\begin{CSLReferences}{0}{1}
\end{CSLReferences}



\end{document}
